% Options for packages loaded elsewhere
\PassOptionsToPackage{unicode}{hyperref}
\PassOptionsToPackage{hyphens}{url}
%
\documentclass[
  11pt,
]{article}
\usepackage{lmodern}
\usepackage{amssymb,amsmath}
\usepackage{ifxetex,ifluatex}
\ifnum 0\ifxetex 1\fi\ifluatex 1\fi=0 % if pdftex
  \usepackage[T1]{fontenc}
  \usepackage[utf8]{inputenc}
  \usepackage{textcomp} % provide euro and other symbols
\else % if luatex or xetex
  \usepackage{unicode-math}
  \defaultfontfeatures{Scale=MatchLowercase}
  \defaultfontfeatures[\rmfamily]{Ligatures=TeX,Scale=1}
\fi
% Use upquote if available, for straight quotes in verbatim environments
\IfFileExists{upquote.sty}{\usepackage{upquote}}{}
\IfFileExists{microtype.sty}{% use microtype if available
  \usepackage[]{microtype}
  \UseMicrotypeSet[protrusion]{basicmath} % disable protrusion for tt fonts
}{}
\makeatletter
\@ifundefined{KOMAClassName}{% if non-KOMA class
  \IfFileExists{parskip.sty}{%
    \usepackage{parskip}
  }{% else
    \setlength{\parindent}{0pt}
    \setlength{\parskip}{6pt plus 2pt minus 1pt}}
}{% if KOMA class
  \KOMAoptions{parskip=half}}
\makeatother
\usepackage{xcolor}
\IfFileExists{xurl.sty}{\usepackage{xurl}}{} % add URL line breaks if available
\IfFileExists{bookmark.sty}{\usepackage{bookmark}}{\usepackage{hyperref}}
\hypersetup{
  hidelinks,
  pdfcreator={LaTeX via pandoc}}
\urlstyle{same} % disable monospaced font for URLs
\usepackage[a4paper,margin=2.5cm]{geometry}
\usepackage{color}
\usepackage{fancyvrb}
\newcommand{\VerbBar}{|}
\newcommand{\VERB}{\Verb[commandchars=\\\{\}]}
\DefineVerbatimEnvironment{Highlighting}{Verbatim}{commandchars=\\\{\}}
% Add ',fontsize=\small' for more characters per line
\newenvironment{Shaded}{}{}
\newcommand{\AlertTok}[1]{\textcolor[rgb]{1.00,0.00,0.00}{\textbf{#1}}}
\newcommand{\AnnotationTok}[1]{\textcolor[rgb]{0.38,0.63,0.69}{\textbf{\textit{#1}}}}
\newcommand{\AttributeTok}[1]{\textcolor[rgb]{0.49,0.56,0.16}{#1}}
\newcommand{\BaseNTok}[1]{\textcolor[rgb]{0.25,0.63,0.44}{#1}}
\newcommand{\BuiltInTok}[1]{#1}
\newcommand{\CharTok}[1]{\textcolor[rgb]{0.25,0.44,0.63}{#1}}
\newcommand{\CommentTok}[1]{\textcolor[rgb]{0.38,0.63,0.69}{\textit{#1}}}
\newcommand{\CommentVarTok}[1]{\textcolor[rgb]{0.38,0.63,0.69}{\textbf{\textit{#1}}}}
\newcommand{\ConstantTok}[1]{\textcolor[rgb]{0.53,0.00,0.00}{#1}}
\newcommand{\ControlFlowTok}[1]{\textcolor[rgb]{0.00,0.44,0.13}{\textbf{#1}}}
\newcommand{\DataTypeTok}[1]{\textcolor[rgb]{0.56,0.13,0.00}{#1}}
\newcommand{\DecValTok}[1]{\textcolor[rgb]{0.25,0.63,0.44}{#1}}
\newcommand{\DocumentationTok}[1]{\textcolor[rgb]{0.73,0.13,0.13}{\textit{#1}}}
\newcommand{\ErrorTok}[1]{\textcolor[rgb]{1.00,0.00,0.00}{\textbf{#1}}}
\newcommand{\ExtensionTok}[1]{#1}
\newcommand{\FloatTok}[1]{\textcolor[rgb]{0.25,0.63,0.44}{#1}}
\newcommand{\FunctionTok}[1]{\textcolor[rgb]{0.02,0.16,0.49}{#1}}
\newcommand{\ImportTok}[1]{#1}
\newcommand{\InformationTok}[1]{\textcolor[rgb]{0.38,0.63,0.69}{\textbf{\textit{#1}}}}
\newcommand{\KeywordTok}[1]{\textcolor[rgb]{0.00,0.44,0.13}{\textbf{#1}}}
\newcommand{\NormalTok}[1]{#1}
\newcommand{\OperatorTok}[1]{\textcolor[rgb]{0.40,0.40,0.40}{#1}}
\newcommand{\OtherTok}[1]{\textcolor[rgb]{0.00,0.44,0.13}{#1}}
\newcommand{\PreprocessorTok}[1]{\textcolor[rgb]{0.74,0.48,0.00}{#1}}
\newcommand{\RegionMarkerTok}[1]{#1}
\newcommand{\SpecialCharTok}[1]{\textcolor[rgb]{0.25,0.44,0.63}{#1}}
\newcommand{\SpecialStringTok}[1]{\textcolor[rgb]{0.73,0.40,0.53}{#1}}
\newcommand{\StringTok}[1]{\textcolor[rgb]{0.25,0.44,0.63}{#1}}
\newcommand{\VariableTok}[1]{\textcolor[rgb]{0.10,0.09,0.49}{#1}}
\newcommand{\VerbatimStringTok}[1]{\textcolor[rgb]{0.25,0.44,0.63}{#1}}
\newcommand{\WarningTok}[1]{\textcolor[rgb]{0.38,0.63,0.69}{\textbf{\textit{#1}}}}
\setlength{\emergencystretch}{3em} % prevent overfull lines
\providecommand{\tightlist}{%
  \setlength{\itemsep}{0pt}\setlength{\parskip}{0pt}}
\setcounter{secnumdepth}{-\maxdimen} % remove section numbering
\usepackage{amsmath, amssymb, amsthm, physics, mathtools}
\usepackage[utf8]{inputenc}
\usepackage{xcolor}
\usepackage{float}
\usepackage{parskip}
\raggedbottom
\setlength{\parskip}{0.8em}
\setlength{\parindent}{0em}
\linespread{1.1}
\usepackage{titlesec}
\titleformat{\chapter}{\normalfont\huge\bfseries}{\thechapter}{1em}{}
\titleformat{\section}{\normalfont\Large\bfseries}{\thesection}{1em}{}
\usepackage{tikz, pgfplots}
\pgfplotsset{compat=newest}
\usepackage{listings}
\lstset{basicstyle=\ttfamily\small, breaklines=true, frame=single, language=C++, keywordstyle=\color{blue}, commentstyle=\color{green!60!black}}
\usepackage{hyperref}
\hypersetup{colorlinks=true, linkcolor=blue, urlcolor=blue}

\author{}
\date{}

\begin{document}

\begin{center}
    \LARGE\bfseries Adaptive Finite Elements for a Linear Parabolic Problem
\end{center}
\vspace{2em}

\section{1. Introduction}

Partial Differential Equations (PDEs) are fundamental mathematical
models for phenomena across various scientific and engineering
disciplines. While analytical solutions are often intractable for
complex problems, numerical methods, particularly the Finite Element
Method (FEM), offer robust approximation strategies. For time-dependent
problems, such as the heat equation, the computational domain spans both
space and time, presenting challenges in achieving accuracy and
efficiency with uniform discretizations.

Adaptive methods represent a significant advancement in numerical
simulation, allowing for dynamic adjustment of spatial and temporal
resolutions based on estimates of the solution error. This targeted
refinement concentrates computational resources where they are most
needed, for example, near sharp fronts, singularities, or regions of
high solution gradients, leading to more accurate results with optimized
computational cost compared to global uniform meshes \cite{picasso1998}.

This theoretical study focuses on the linear parabolic problem,
specifically the heat equation, with an emphasis on its strong and weak
formulations, mathematical analysis, numerical discretization via FEM
coupled with time-stepping schemes, and the theoretical underpinnings of
error estimation crucial for adaptive algorithms. The goal is to lay the
groundwork for developing an adaptive formulation that dynamically refines
the space-time grid based on error indicators, a strategy extensively
discussed in the context of a posteriori error estimates for parabolic
problems \cite{picasso1998}.

\section{2. Mathematical Formulation}

We consider a linear parabolic problem, often referred to as the heat
equation, defined on a spatial domain $\Omega \subset \mathbb{R}^d$
(typically $d=2$ or $d=3$) over a time interval $(0, T)$. The
problem describes the evolution of an unknown function $u(x, t)$ that
depends on both spatial coordinates $x$ and time $t$.

\subsection{Strong Formulation}

The strong form of the governing equation is given by:

\[
\frac{\partial u}{\partial t} - \nabla \cdot (\mu \nabla u) = f \quad \text{in } \Omega \times (0, T) \quad (P.1)
\]

The spatial domain $\Omega$ is assumed to be a convex polygon of
$\mathbb{R}^2$ \cite{picasso1998}. The problem is complemented by the following
auxiliary conditions:

\begin{itemize}
\item
  \textbf{Boundary Conditions (BCs):} Homogeneous Neumann conditions are
  applied over the entire spatial boundary $\partial \Omega$ for all
  $t \in (0, T)$: \[
  \mu \nabla u \cdot n = 0 \quad \text{on } \partial\Omega \times (0, T) \quad (P.2)
  \] Here, $n$ is the outward unit normal vector to
  $\partial\Omega$. This condition implies no heat flux across the
  boundary.
\item
  \textbf{Initial Condition (IC):} The initial state of the system is
  prescribed as zero throughout the domain $\Omega$ at $t=0$: \[
  u(x, 0) = 0 \quad \text{in } \Omega \times \{0\} \quad (P.3)
  \]
\end{itemize}

The parameters and forcing term are defined as follows:

\begin{itemize}
\tightlist
\item
  $\mu$: Diffusion coefficient, which is constant and set to
  $\mu = 1$.
\item
  $f(x, t)$: Forcing term, representing a sequence of impulses
  localized in space and time. It is defined as a separable function of
  space and time: \[
  f(x, t) = g(t)h(x)
  \] where \[
  g(t) = \frac{\exp(-a \cos(2 N \pi t))}{\exp(a)}
  \] and \[
  h(x) = \exp\left(-\frac{(x - x_0)^2}{\sigma^2}\right)
  \] with $a > 0$, $N \in \mathbb{N}$, $x_0 \in \Omega$, and
  $\sigma > 0$.
\end{itemize}

The overall computational domain is the space-time cylinder
$Q_T = \Omega \times (0, T]$.

\subsection{Weak Formulation}

To enable numerical solution, especially for solutions that may not
possess the high regularity required for the strong form, we derive the
weak (variational) formulation of the problem. This involves multiplying
the PDE by a spatially-dependent test function and integrating over the
spatial domain.

Let $V$ be a suitable functional space for the solution $u$ and test
functions $v$. We seek $u(t, \cdot) \in V$ such that for all
$v \in V$, the integral form of the PDE is satisfied. The space $V$
will typically be a Sobolev space, as discussed below.

\begin{enumerate}
\def\labelenumi{\arabic{enumi}.}
\item
  \textbf{Multiply by a Test Function and Integrate:} Multiply equation
  (P.1) by a test function $v(x) \in V$ and integrate over $\Omega$:
  \[
  \int_{\Omega} \left( \frac{\partial u}{\partial t} - \nabla \cdot (\mu \nabla u) \right) v \, dx = \int_{\Omega} f v \, dx \quad \forall v \in V, \text{ a.e. } t \in (0, T)
  \] This can be split into two terms: \[
  \int_{\Omega} \frac{\partial u}{\partial t} v \, dx - \int_{\Omega} \nabla \cdot (\mu \nabla u) v \, dx = \int_{\Omega} f v \, dx
  \]
\item
  \textbf{Apply Green's First Identity (Integration by Parts):} The
  second term on the left-hand side involves a divergence. We apply
  Green's First Identity (also known as the Divergence Theorem) to this
  term: \[
  - \int_{\Omega} \nabla \cdot (\mu \nabla u) v \, dx = \int_{\Omega} \mu \nabla u \cdot \nabla v \, dx - \int_{\partial \Omega} (\mu \nabla u \cdot n) v \, ds
  \]
\item
  \textbf{Incorporate Boundary Conditions:} The boundary integral term
  is simplified by the given homogeneous Neumann boundary condition
  (P.2): \[
  \int_{\partial \Omega} (\mu \nabla u \cdot n) v \, ds = \int_{\partial \Omega} (0) v \, ds = 0
  \] Thus, the boundary integral vanishes. This implies that the Neumann
  boundary conditions are ``natural'' boundary conditions in the
  variational formulation, as they are implicitly satisfied without
  further restrictions on the test function space.
\item
  \textbf{The Weak Formulation:} Combining these steps, the
  time-continuous weak formulation of the problem is: Find
  $u: (0, T) \to V$ such that $u(x,0)=0$ and for all $v \in V$: \[
  \int_{\Omega} \frac{\partial u}{\partial t} v \, dx + \int_{\Omega} \mu \nabla u \cdot \nabla v \, dx = \int_{\Omega} f v \, dx \quad \forall v \in V, \text{ a.e. } t \in (0, T) \quad (W.1)
  \]
\item
  \textbf{Functional Spaces:} For the terms in (W.1) to be well-defined,
  $u$ and $v$ must have square-integrable first spatial derivatives.
  The standard choice for such problems is the Sobolev space
  $H^1(\Omega)$. Since there are no Dirichlet boundary conditions
  constraining the test function values to zero on the boundary, the
  test space $V$ is simply $H^1(\Omega)$.

  More precisely, the solution $u$ is sought in the space
  $W = \{w \in L^2(0,T;H^1(\Omega)) \text{ and } \partial w/\partial t\in L^2(0,T;H^{-1}(\Omega))\}$
  \cite{picasso1998}. The initial condition $u(\cdot,0)=u^0$ is understood in
  the $L^2(\Omega)$ sense if
  $u \in \mathcal{C}^0([0,T];L^2(\Omega))$, which is a property of
  functions in $W$ \cite{picasso1998}.

  We can express (W.1) using the mass matrix-like and stiffness
  matrix-like terms: \[
  \left( \frac{\partial u}{\partial t}, v \right)_{L^2(\Omega)} + A(u, v) = (f, v)_{L^2(\Omega)} \quad \forall v \in V, \text{ a.e. } t \in (0, T)
  \] where $A(u, v) = \int_{\Omega} \mu \nabla u \cdot \nabla v \, dx$
  is the bilinear form and
  $(f, v)_{L^2(\Omega)} = \int_{\Omega} f v \, dx$ is the linear
  functional.
\end{enumerate}

\subsection{Mathematical Analysis}

The mathematical analysis for parabolic problems involves establishing
the existence, uniqueness, and stability of the weak solution. These
properties are critical to guarantee that the problem is well-posed and
that numerical approximations are meaningful. The analysis often draws
on functional analysis tools, such as the Lax-Milgram Lemma for the
spatial elliptic operator, extended to the time-dependent setting via
energy methods.

\begin{enumerate}
\def\labelenumi{\arabic{enumi}.}
\item
  \textbf{Continuity of the Bilinear Form $A(u,v)$:} The bilinear form
  $A(u,v) = \int_{\Omega} \mu \nabla u \cdot \nabla v \, dx$ must be
  continuous (bounded) on $V \times V$. For $V = H^1(\Omega)$, using
  the Cauchy-Schwarz inequality and assuming $\mu$ is bounded (here
  $\mu=1$): \[
  |A(u, v)| = \left| \int_{\Omega} \mu \nabla u \cdot \nabla v \, dx \right| \le \mu_{\max} \int_{\Omega} |\nabla u| |\nabla v| \, dx \le \mu_{\max} \|\nabla u\|_{L^2(\Omega)} \|\nabla v\|_{L^2(\Omega)}
  \] Since $\|\nabla u\|_{L^2(\Omega)} \le \|u\|_{H^1(\Omega)}$ (where
  $\|u\|_{H^1(\Omega)}^2 = \|u\|_{L^2(\Omega)}^2 + \|\nabla u\|_{L^2(\Omega)}^2$),
  we have: \[
  |A(u, v)| \le \mu_{\max} \|u\|_{H^1(\Omega)} \|v\|_{H^1(\Omega)}
  \] This proves continuity with $M = \mu_{\max}$ (here $M=1$).
\item
  \textbf{Coercivity of the Bilinear Form $A(v,v)$:} The bilinear form
  $A(v,v)$ must be coercive. For the pure diffusion term
  $A(v, v) = \int_{\Omega} \mu |\nabla v|^2 \, dx$, with homogeneous
  Neumann boundary conditions on the entire boundary, $A(v,v)$ only
  involves the gradient term. This means $A(v,v) = 0$ for any constant
  function $v \ne 0$. Therefore, $A(v,v)$ is not coercive with
  respect to the full $H^1(\Omega)$ norm directly.

  However, for parabolic problems, a weaker coercivity condition is
  often sufficient, usually called \textbf{weak coercivity}. This states
  that there exists a constant $\lambda \ge 0$ such that
  $A(V, V) + \lambda \|V\|_{L^2(\Omega)}^2 \ge \alpha \|V\|_V^2$ for
  some $\alpha > 0$. This modified form effectively makes the bilinear form coercive
  on $H^1(\Omega)$. The non-trivial $L^2$ term (the time derivative)
  in the parabolic equation helps establish well-posedness even when the
  spatial operator is only weakly coercive.
\item
  \textbf{Boundedness of the Linear Functional $F(v)$:} The linear
  functional $F(v) = \int_{\Omega} f v \, dx$ must be bounded
  (continuous) on $V$. Using the Cauchy-Schwarz inequality: \[
  |F(v)| = \left| \int_{\Omega} f v \, dx \right| \le \|f\|_{L^2(\Omega)} \|v\|_{L^2(\Omega)}
  \] Since $\|v\|_{L^2(\Omega)} \le \|v\|_{H^1(\Omega)}$, we have: \[
  |F(v)| \le \|f\|_{L^2(\Omega)} \|v\|_{H^1(\Omega)}
  \] This proves boundedness with $C_F = \|f\|_{L^2(\Omega)}$.
\item
  \textbf{Existence, Uniqueness, and Stability:} For the parabolic heat
  equation with an $L^2$ initial condition and $L^2$ forcing, it is
  a standard result that a unique weak solution exists in the space
  $W = \{w\in L^2(0,T;H^1(\Omega)) \text{ and } \partial w/\partial t\in L^2(0,T;H^{-1}(\Omega))\}$
  \cite{picasso1998}. Moreover, the solution $u(\cdot,t)$ belongs to
  $\mathcal{C}^0([0,T];L^2(\Omega))$ \cite{picasso1998}.

  The stability of the solution is demonstrated through energy
  estimates. For the semi-discrete problem, the solution norm is bounded
  by the data. If $F=0$, the $L^2$ norm of the solution at time
  $t$ is bounded by its initial condition, and the time-integrated
  $H^1$ semi-norm is also bounded: \[
  \|U_h(t)\|_{L^2(\Omega)}^2 + \alpha \int_0^t \|\nabla U_h(\tau)\|_{L^2(\Omega)}^2 d\tau \le \|U_0^h\|_{L^2(\Omega)}^2
  \] This implies that the solution is stable and does not grow
  unboundedly.
\end{enumerate}

\section{3. Numerical Discretization}

The weak formulation (W.1) still represents a continuous problem in both
space and time. To obtain a computable solution, both spatial and
temporal dimensions must be discretized.

\subsection{Spatial Semi-Discretization: Finite Element Method (FEM)}

We discretize the spatial domain $\Omega$ using FEM.
\begin{enumerate}
\item \textbf{Finite Element Space:} We introduce a finite-dimensional subspace
$V_h \subset V$ spanned by a basis of continuous, piecewise polynomial
functions $\{\phi_j(x)\}_{j=1}^{N_h}$. We employ continuous, piecewise linear triangular finite elements
($P_1$ elements) \cite{picasso1998}.
\item \textbf{Approximate Solution:}
The solution $u(x,t)$ is approximated by $u_h(x,t)$ in $V_h$: \[
    u_h(x, t) = \sum_{j=1}^{N_h} U_j(t) \phi_j(x)
    \] where $U_j(t)$ are time-dependent coefficients.
\item \textbf{Galerkin Procedure:} Substituting $u_h$ into (W.1) and
choosing test functions $v = \phi_i(x)$ for each $i=1, \dots, N_h$,
we obtain a system of Ordinary Differential Equations (ODEs) in time,
known as the semi-discrete system: \[
    \sum_{j=1}^{N_h} \frac{d U_j}{d t} \int_{\Omega} \phi_j \phi_i \, dx + \sum_{j=1}^{N_h} U_j(t) \int_{\Omega} \mu \nabla \phi_j \cdot \nabla \phi_i \, dx = \int_{\Omega} f \phi_i \, dx
    \] This system is expressed in matrix form: \[
    M \frac{d\mathbf{U}}{dt} + A \mathbf{U}(t) = \mathbf{F}(t) \quad (SD.1)
    \] where $\mathbf{U}(t) = [U_1(t), \dots, U_{N_h}(t)]^T$ is the
vector of nodal unknowns, and:
\begin{itemize}
\item \textbf{Mass Matrix $M$}: $M_{ij} = \int_{\Omega} \phi_j \phi_i \, dx$.
\item \textbf{Stiffness Matrix $A$}: $A_{ij} = \int_{\Omega} \mu \nabla \phi_j \cdot \nabla \phi_i \, dx$.
\item \textbf{Load Vector $\mathbf{F}(t)$}: $F_i(t) = \int_{\Omega} f(x,t) \phi_i \, dx$.
\end{itemize}
\end{enumerate}

The initial condition for the ODE system is
$\mathbf{U}(0)=\mathbf{0}$, obtained by interpolating $u(x,0)=0$.

\subsection{Temporal Discretization: Theta Method}

To solve the semi-discrete system (SD.1), we discretize the time
variable. We introduce a partition of $(0,T)$ into subintervals
$(t^n-t^{n-1})$ and define time levels
$0=t_0 < t_1 < \dots < t_N=T$. Let $\Delta t_n = t_n - t_{n-1}$ be
the time step. We seek an approximation
$\mathbf{U}^n \approx \mathbf{U}(t^n)$.

The \textbf{Theta-method} is a one-step finite difference scheme that
approximates the time derivative and weights the spatial terms between
$t^n$ and $t^{n+1}$. The scheme for advancing from $\mathbf{U}^n$ to
$\mathbf{U}^{n+1}$ is: \[
M \left( \frac{\mathbf{U}^{n+1} - \mathbf{U}^n}{\Delta t} \right) + A \left[ \theta \mathbf{U}^{n+1} + (1-\theta) \mathbf{U}^n \right] = \theta \mathbf{F}^{n+1} + (1-\theta) \mathbf{F}^n \quad (FD.1)
\] where $\mathbf{F}^{n+1} = \mathbf{F}(t^{n+1})$ and
$\mathbf{F}^n = \mathbf{F}(t^n)$.

Rearranging to solve for $\mathbf{U}^{n+1}$ at each time step, we get
a linear algebraic system: \[
\mathbf{K} \mathbf{U}^{n+1} = \mathbf{R}^n
\] where \[
\mathbf{K} = \frac{1}{\Delta t} M + \theta A \quad (FD.2)
\] and \[
\mathbf{R}^n = \left( \frac{1}{\Delta t} M - (1-\theta)A \right) \mathbf{U}^n + \theta \mathbf{F}^{n+1} + (1-\theta)\mathbf{F}^n \quad (FD.3)
\]

Following \cite{picasso1998}, we utilize the \textbf{Euler implicit (Backward Euler) scheme} for time
discretization. This corresponds to choosing $\theta = 1$.

For $\theta=1$:
\begin{itemize}
\item The scheme is first-order accurate in time ($O(\Delta t)$).
\item It is unconditionally stable, meaning it remains bounded for any choice of $\Delta t > 0$.
\end{itemize}

The fully discrete scheme for Backward Euler is: \[
\left( \frac{1}{\Delta t} M + A \right) \mathbf{U}^{n+1} = \frac{1}{\Delta t} M \mathbf{U}^n + \mathbf{F}^{n+1} \quad (FD.4)
\] At each time step, a linear system with matrix
$K = \frac{1}{\Delta t} M + A$ is solved for $\mathbf{U}^{n+1}$.
Since $M$ and $A$ are constant (as $\mu$ is constant), $K$ only
needs to be assembled and factorized once if $\Delta t$ is fixed,
significantly reducing computational cost.

\subsection{Theoretical Error Estimates}

The accuracy of the fully discrete solution $u_{h\tau}(x,t)$ (where
$h$ denotes spatial discretization and $\tau$ denotes temporal
discretization) is determined by both the spatial and temporal errors.

For a linear parabolic problem solved with FEM of polynomial degree
$r$ in space and a $\theta$-method of order $Q(\theta)$ in time,
the error is typically bounded by: \[
\|u(t_n) - u_{h\tau}(t_n)\|_{norm} \le C \left( H^r + (\Delta t)^{Q(\theta)} \right) \quad (\text{Error.1})
\] where $H$ is the mesh size, and $C$ is a constant dependent on
the exact solution's regularity but independent of $H$ and
$\Delta t$.

Given the proposed scheme (piecewise linear FEM, i.e., $r=1$, and
Backward Euler, i.e., $Q(\theta)=1$ for $\theta \neq 1/2$), the
theoretical error estimate for the exact solution $u$ and the
numerical approximation $u_{h\tau}$ would be: \[
\|u - u_{h\tau}\|_{norm} \le C (H + \Delta t)
\] This indicates first-order convergence in both space and time.

Detailed a posteriori error estimates for the error $e = u - u_{h\tau}$ in the $L^2$ in time, $H^1$ in space norm are established in \cite{picasso1998}. The error estimator is comprised of three main contributions: $\eta$, $\epsilon$, and $\gamma$ \cite{picasso1998}.

\begin{enumerate}
\def\labelenumi{\arabic{enumi}.}
\item
  \textbf{Spatial Residual Error ($\eta^n$):} This term measures the
  residual of the PDE equation in each element $K$ and jump of the
  normal derivative across edges $l$: \[
  (\eta_{K}^{n})^{2}=\int_{t^{n-1}}^{t^{n}}\{|K|\|\Pi_{K}^{n}(f-\frac{\partial u_{h\tau}}{\partial t})\|_{0,K}^{2}+\frac{1}{2}\sum_{l\in E_{K}^{n}}|l|\|\Pi_{l}^{n}(J_{l}^{n})\|_{0,l}^{2}\}dt \quad (\text{Error.2})
  \] This is similar to error estimators for elliptic problems,
  capturing the spatial discretization error, particularly due to the
  piecewise linear approximation and local non-satisfaction of the PDE
  \cite{picasso1998}.
\item
  \textbf{Temporal Error from Solution Difference ($\epsilon^n$):}
  This term estimates the error introduced by the time discretization
  scheme based on the difference between solutions at consecutive time
  steps: \[
  (\epsilon_{K}^{n})^{2}=\int_{t^{n-1}}^{t^{n}}\|\nabla(u_{h}^{n}-u_{h}^{n-1})\|_{0,K}^{2}dt \quad (\text{Error.3})
  \] This component measures the effect of large solution changes
  between time levels \cite{picasso1998}.
\item
  \textbf{Temporal Error from Forcing Term Approximation
  ($\gamma^n$):} This term quantifies the error due to approximating
  the time-dependent forcing function $f$ with its value at $t^n$
  (or some average): \[
  (\gamma_{K}^{n})^{2}=\int_{t^{n-1}}^{t^{n}}\|f-f^{n}\|_{0,K}^{2}dt \quad (\text{Error.4})
  \] Here $f^n=f(\cdot,t^n)$ \cite{picasso1998}. This captures the error from
  not accurately resolving the time-dependence of the source term.
\end{enumerate}

The total error estimators $\eta, \epsilon, \gamma$ are summed over all
elements and time steps \cite{picasso1998}.

\textbf{Upper and Lower Bounds:} Theoretical bounds on the error $e=u-u_{h\tau}$ are provided in \cite{picasso1998}. 
\begin{itemize}
\item
    \textbf{Upper Bound (Theorem 2.1):} Under certain assumptions (regular
    meshes, nested triangulations, $f \in H^1((0,T)\times\Omega)$), for
    $h$ small enough, the error is bounded above: \[
    \|e(\cdot,t^{n})\|_{0,\Omega}^{2}+\int_{t^{n-1}}^{t^{n}}\|\nabla e\|_{0,\Omega}^{2}dt\le\|e(\cdot,t^{n-1})\|_{0,\Omega}^{2}+C((\eta^{n})^{2}+(\epsilon^{n})^{2}+(\gamma^{n})^{2}+h^{4}|f|_{1,(t^{n-1},t^{n})\times\Omega}^{2}) \quad (\text{Error.5})
    \] Summing over time steps, the global error follows
    $\int_{0}^{T}\|\nabla e\|_{0,\Omega}^{2}dt\le C(\eta^{2}+\epsilon^{2}+\gamma^{2})$
    \cite{picasso1998}. 
\item \textbf{Lower Bound (Theorem 2.2):} Conversely, the
    error estimator itself is bounded by the true error, confirming its
    reliability: \[
    (\eta^{n})^{2}\le C(\int_{t^{n-1}}^{t^{n}}\|\nabla e\|_{0,\Omega}^{2}dt+(\epsilon^{n})^{2}+h^{4}|f|_{1,(t^{n-1},t^{n})\times\Omega}^{2}) \quad (\text{Error.6})
    \] And globally,
    $\eta^{2}\le C(\int_{0}^{T}\|\nabla e\|_{0,\Omega}^{2}dt+\epsilon^{2})$
    \cite{picasso1998}.
\end{itemize}

These bounds demonstrate that the introduced error estimator terms
effectively capture the overall discretization error in both space and
time. This framework is essential for adaptive algorithms, as it
provides a computable measure of the error for refining the mesh and
adjusting the time step \cite{picasso1998}.

\section{4. Implementation Details}

% Questa sezione è lasciata intenzionalmente vuota per i futuri dettagli di implementazione del codice.

\section{5. Numerical Experiments}

% Questa sezione è lasciata intenzionalmente vuota per descrivere la configurazione dei test.

\section{6. Results}

% Questa sezione è lasciata intenzionalmente vuota per l'analisi dei risultati.

\section{7. References}

\begin{thebibliography}{99}
\bibitem{picasso1998}
M. Picasso, ``Adaptive finite elements for a linear parabolic problem,'' \emph{Comput. Methods Appl. Mech. Engrg.}, vol. 167, pp. 223--237, 1998. (Département de Mathématiques, Ecole Polytechnique Fédérale de Lausanne).
\end{thebibliography}

\end{document}